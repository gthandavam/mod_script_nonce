\documentclass{article}
\usepackage{graphicx}
\usepackage{fullpage}
\usepackage{algorithm}
\usepackage{algpseudocode}


\newcommand{\headings}[2]{
\begin{center}
\bf CSE508: Network Security		\\
\bf Project Report					\\
\bf Script Nonce Module for Apache	\\
\end{center}
{\centerline{ { Ameya Hemant Patnekar, Ganesa Thandavam Ponnuraj, Lalit Ramesh Sirsikar } } }
{{\bf Date:} #2} \hfill {{\bf Professor:} #1} \\
\rule[0.1in]{\textwidth}{0.020in}
}


\begin{document}
\headings{Prof. Rob Johnson}{May-23,2013}

\noindent
\textbf{Introduction:} 
\medskip

Cross-site scripting (XSS) is a type of computer security vulnerability typically found in Web applications. XSS enables attackers to inject client-side script into Web pages viewed by other users. A cross-site scripting vulnerability may be used by attackers to bypass access controls such as the same origin policy.
\\
\indent
Generally, two types of attacks are possible, reflected and stored XSS attacks. One of the example for reflected attacks occur when a user clicks on a malicious URL, which gets directed to the server; the server accepts the URL supposing the user as genuine and responds back. The response carries with itself the malicious scripts which then get executed on the user's browser. The browser also accepts this based on the same origin policy security concept.
An example of a stored XSS attack is as follows: Suppose there is a comment section on some website. Now if a user injects a script in this text-field, it then goes to the server and gets stored in the Data base. When some other user logs on to that page again, the script gets loaded as a part of page. And as the malicious script is coming from the server, the victims browser executes it due to same origin policy and thus the attack is accomplished.

This report first states the problem, and then proposes solution to the problem. Our implementation is then described in detail followed by benchmarking results.
%%%%%%%%%%%%%%%%%%%%%%%%%%%%%%%%%
\\

\noindent
\textbf{Problem Description:} 
\medskip

The problem with the above attacks is that malicious scripts can get executed on client's browser. There are specific methods to control this from client's side, but a more general scene would be to tackle these attacks from the server side itself. The server attempts to prevent this attack from its side by inserting nonces into script values (Content Security Policy script nonce). The client then executes only those scripts that have valid nonces in them. Here, our implementation is described to demonstrate this on Apache server.
%%%%%%%%%%%%%%%%%%%%%%%%%%%%%%%%%
\\%skip

\noindent
\textbf{Implementation:} 
\medskip

The implementation to prevent XSS attacks on the server side has been made as follows. Apache comes with a module called $mod\_substitute$ which when configured and stated in httpd.conf allows for replacement of a string with another. The syntax of doing that is as follows:
\\
\begin{algorithmic}
\State {\textsf{ \textless Location /\textgreater AddOutputFilterByType SUBSTITUTE text/html}}
\State {\textsf{Substitute s/foo/bar/ni}}
\State {\textsf{\textless /Location \textgreater}}
\end{algorithmic}
%\medskip

which substitutes each occurrence of string "foo" with string "bar" and it also accepts any optional parameters that the user wants to give - 'n' stands for treating the pattern as a fixed string as opposed to a regular expression, and 'i' performs a case-insensitive match. Exploiting this feature of $mod\_substitute$, we have defined our own module and loaded it by specifying in httpd.conf file. The syntax of our module is as follows:
\\
\begin{algorithmic}
\State {\textsf{\textless Location /\textgreater AddOutputFilterByType ADD-SCRIPT-NONCE text/html}}
\State {\textsf {Addscriptnonce s/server-secret/}}
\State {\textsf{\textless /Location \textgreater}}
\end{algorithmic}
\medskip

\noindent
\textit{Algorithm: } We crawl through the entire website with a python module which takes as input a set of file-names and replaces each occurrence of string "\textless script" with "\textless script nonce=server-secret", thus each page now has a script-nonce attached with it, which at present is servers secret value. This value for nonce during pre-processing is configurable, and can be modified periodically by the web-admin.

\noindent
Next, our module is enabled through httpd.conf by specifying the following:
%\medskip

\noindent
\textsf{ LoadModule script\_nonce\_module modules$/$mod\_script\_nonce.so}
%\medskip

The module acts by attaching a script-nonce header field in the response URL and filling it with random nonce value (we use $apr\_generate\_random\_bytes$ for this and convert all hexadecimal values to alphanumeric characters). The module at the same time also fills all the occurrences of the server-secret given in httpd.conf with the value generated. The page when delivered to client (that understands CSP1.1 nonce values) checks if the nonce value matches; the script is executed if the nonce value matches, else it is not executed.

%%%%%%%%%%%%%%%%%%%%%%%%%%%%%%%%%%%%%%%%%%%%%%%%%%%%%%55
\bigskip

\noindent
\textbf{Defence:}
%\medskip

\noindent Stored XSS:
%\medskip

Every valid script delivered by the server contains a script nonce with a random value. And the same nonce  value is also specified in the CSP Header. So, even if any attacker succeeds to inject his malicious script into the page, it won't be executed by the browser as it will not contain valid script nonce.

%\bigskip
\noindent Reflected XSS:
%\medskip

Chrome uses an Anti-XSS filter, based on static analysis, which attempts to detect XSS. If it detects 
such an attempt, it filters out the injected code, and effectively stops the on-going attack.
We tested the following specially crafted URL which contain Script in it. Although it worked with Internet explorer and Mozila firefox, it didn't work with Chrome due to Chrome's Anti-XSS filter. Here are some examples:
%\medskip

http://securitee.tk/files/chrome\_xss.php?
a=$<$script$>$alert(1);\&b=bar

http://securitee.tk/files/chrome\_xss.php?
a=$<$script$>$alert(1);/*\&b=*/$<$/script$>$

http://securitee.tk/files/chrome\_xss.php?
a=$<$script$>$/*//\&b=*/alert(1);$<$/script$>$

http://securitee.tk/files/chrome\_xss.php?
a=$<$script$>$void('\&b=');alert(1);$<$/script$>$
%%%%%%%%%%%%%%%%%%%%%%%%%%%%%%%%%

\bigskip
\noindent
\textbf{Performance:} 
\medskip

The performance was measured with Apache's benchmarking tool called 'ab'. The tool comes as a module with Apache which takes as input 3 parameters - total number of requests, number of concurrent requests and URL. It then gives a complete output stating various time values (Connect, Processing, Waiting, Total).
We wrote a python script for testing mean value results for a page containing 80 script-nonce tags. The module runs the command "ab -n 1000 -c 50+50*i http://www.sirsikar.com/benchmarkResults" for i in range(1,11) and takes the mean over all the values obtained. The number of iterations can be user-specified. The data here is for 5 iterations:

\begin{table}[ht]            % title of Table
%\centering                                   % used for centering table
\begin{tabular}{|p{1cm} |p{3cm} |p{3cm} |p{3cm} |p{3cm} |} % centered columns (4 columns)
%\textbf{Response time for longest request:}
\hline                               %inserts double horizontal lines
Case & Total Number Of Request & Number of Simultaneous Requests & Number of requests per second Without Nonce & Number of requests per second With Nonce \\ [0.5ex] % inserts table 
%heading
\hline % inserts single horizontal line
1 & 1000 & 50 & 3115.37 & 331.248 \\
2 & 1000 & 100 & 3132.208 & 315.944 \\
3 & 1000 & 150 & 3172.228 & 327.4 \\
4 & 1000 & 200 & 3117.508 & 339.14 \\
5 & 1000 & 250 & 3150.548 & 337.932 \\
6 & 1000 & 300 & 3114.326 & 339.43 \\
7 & 1000 & 350 & 3153.028 & 337.454 \\
8 & 1000 & 400 & 3109.476 & 336.076 \\
9 & 1000 & 450 & 3145.006 & 339.49 \\
10 & 1000 & 500 & 3126.9 & 340.498 \\

%[1ex] % [1ex] adds vertical space
\hline %inserts single line
\end{tabular}
\label{table:nonlin} % is used to refer this table in the text
\end{table}
%%%%%%%%%%%%%%%%%%%%%%%%%%%%%%%%%%%%%%%%%%%%%%%%%%%%%%55
\bigskip

\begin{center}
\includegraphics[angle=0,width=140mm]{benchmarkResults.jpg}
\end{center}



\noindent
\textbf{References:}
\medskip

http://www.owasp.org/

http://www.w3.org/

http://www.apache.org/

http://httpd.apache.org/docs/2.2/mod/mod\_substitute.html

http://httpd.apache.org/docs/2.2/programs/ab.html

http://www.cyberciti.biz/tips/howto-performance-benchmarks-a-web-server.html

http://www.ibm.com/developerworks/library/wa-secxss/

http://blog.securitee.org/?p=37



\end{document}

